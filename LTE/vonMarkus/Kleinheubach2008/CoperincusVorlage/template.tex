%% Copernicus Publications Manuscript Preparation Template for LaTeX Submissions
%% ---------------------------------
%% This template should be used for the following class files: copernicus.cls, copernicus2.cls, copernicus_discussions.cls
%% The class files, the Copernicus LaTeX Manual with detailed explanations regarding the comments
%% and some style files are bundled in the Copernicus Latex Package which can be downloaded from the different journal webpages.
%% For further assistance please contact the Publication Production Office (production@copernicus.org).
%% http://publications.copernicus.org


%% Differing comments regarding the specific class files are highlighted.


%% copernicus.cls
\documentclass[ars]{copernicus}

%% copernicus2.cls
%% \documentclass[journal abbreviation]{copernicus2}

%% copernicus_discussions.cls
%% \documentclass[journal abbreviation, hvmath, online]{copernicus_discussions}


\begin{document}


\title{TEXT}


\author[]{NAME}
\author[]{NAME}
\author[]{NAME}

\affil[]{ADDRESS}
\affil[]{ADDRESS}

%% The [] brackets identify the author to the corresponding affiliation, 1, 2, 3, etc. should be inserted.



\runningtitle{TEXT}

\runningauthor{TEXT}

\correspondence{NAME\\ (EMAIL)}



\received{}
\pubdiscuss{} %% only important for two-stage journals
\revised{}
\accepted{}
\published{}

%% These dates will be inserted by the Publication Production Office during the typesetting process.


\firstpage{1}

\maketitle



\begin{abstract}
TEXT
\end{abstract}


%% only used for copernicus2.cls
%% \abstract{
%% TEXT
%% \keywords{TEXT}}



\introduction
%% \introduction[modified heading if necessary]
TEXT



\section{HEADING}
TEXT

\subsection{HEADING}
TEXT

\subsubsection{HEADING}
TEXT




\conclusions
%% \conclusions[modified heading if necessary]
TEXT




%\appendix
%\section{\\ \\ \hspace*{-7mm} HEADING}    %% Appendix A
%
%\subsection                               %% Appendix A1, A2, etc.




\begin{acknowledgements}
TEXT
\end{acknowledgements}




\begin{thebibliography}{}

\bibitem[AUTHOR(YEAR)]{LABEL}
REFERENCE 1

\bibitem[AUTHOR(YEAR)]{LABEL}
REFERENCE 2

...

\end{thebibliography}


%% Literature citations
%% command                        & example result
%% \citet{jones90}|               & Jones et al.\ (1990)
%% \citep{jones90}|               & (Jones et al., 1990)
%% \citep{jones90,jones93}|       & (Jones et al., 1990, 1993)
%% \citep[p.~32]{jones90}|        & (Jones et al., 1990, p.~32)
%% \citep[e.g.,][]{jones90}|      & (e.g., Jones et al., 1990)
%% \citep[e.g.,][p.~32]{jones90}| & (e.g., Jones et al., 1990, p.~32)
%% \citeauthor{jones90}|          & Jones et al.
%% \citeyear{jones90}|            & 1990






%%% FIGURES %%%%%%%%%%%%%%%%%%%%%%%%%%%%%%%%%%%%%%%%%%%%%%%%%%%%%%%%%%%%%%%%%%%%
%
%
%%% ONE-COLUMN FIGURES
%
%%f
%\begin{figure}[t]
%\vspace*{2mm}
%\begin{center}
%\includegraphics[width=8.3cm]{FILE NAME}
%\end{center}
%\caption{TEXT}
%\end{figure}
%
%
%
%%% TWO-COLUMN FIGURES
%
%%f
%\begin{figure*}[t]
%\vspace*{2mm}
%\begin{center}
%\includegraphics[width=12cm]{FILE NAME}
%\end{center}
%\caption{TEXT}
%\end{figure*}
%
%
%%% TABLES %%%%%%%%%%%%%%%%%%%%%%%%%%%%%%%%%%%%%%%%%%%%%%%%%%%%%%%%%%%%%%%%%%%%
%
%
%%% ONE-COLUMN TABLE
%
%%t
%\begin{table}[t]
%\caption{TEXT}
%\vskip4mm
%\centering
%\begin{tabular}{column = lcr}
%\tophline
%
%\middlehline
%
%\bottomhline
%\end{tabular}
%\end{table}
%
%
%%% TWO-COLUMN TABLE
%
%%t
%\begin{table*}[t]
%\caption{TEXT}
%\vskip4mm
%\centering
%\begin{tabular}{column = lcr}
%\tophline
%
%\middlehline
%
%\bottomhline
%\end{tabular}
%\end{table*}
%
%
%%% The different columns must be seperated with a & command and should
%%% end with \\ to identify the column brake.
%
%%%%%%%%%%%%%%%%%%%%%%%%%%%%%%%%%%%%%%%%%%%%%%%%%%%%%%%%%%%%%%%%%%%%%%%%%%%%%%%
%
%
%%% If figures and tables must be numbered 1a, 1b, etc. the following command
%%% should be inserted before the begin{} command.
%
%\addtocounter{figure}{-1}\renewcommand{\thefigure}{\arabic{figure}a}


\end{document}
