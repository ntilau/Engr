%% Copernicus Publications Manuscript Preparation Template for LaTeX Submissions
%% ---------------------------------
%% This template should be used for the following class files: copernicus.cls, copernicus2.cls, copernicus_discussions.cls
%% The class files, the Copernicus LaTeX Manual with detailed explanations regarding the comments
%% and some style files are bundled in the Copernicus Latex Package which can be downloaded from the different journal webpages.
%% For further assistance please contact the Publication Production Office (production@copernicus.org).
%% http://publications.copernicus.org


%% Differing comments regarding the specific class files are highlighted.


%% copernicus.cls
\documentclass[ars]{copernicus}

%% copernicus2.cls
%% \documentclass[journal abbreviation]{copernicus2}

%% copernicus_discussions.cls
%% \documentclass[journal abbreviation, hvmath, online]{copernicus_discussions}


\begin{document}


\title{Efficient Parametric Finite Element Analysis of Passive Microwave Devices Using a Domain Decomposition - Model Order Reduction Approach}


\author[1]{O. Farle}
\author[2]{S. Selleri}
\author[1]{M. L\"osch}
\author[3]{G. Guarnieri}
\author[1]{R. Dyczij-Edlinger}
\author[2]{G. Pelosi}

\affil[1]{Lehrstuhl f\"ur Theoretische Elektrotechnik, Universit\"at des
Saarlandes D-66041 Saarbr\"ucken, Germany}
\affil[2]{Department of Electronics and Telecommunications,
University of Florence, Via C. Lombroso 6/17 - 50134 - Florence, Italy}
\affil[3]{Galileo Avionica S.p.A. - BU Radar Systems Antennas,
via A. Einstein, 35 - 50013 - Campi Bisenzio (FI) Italy}

%% The [] brackets identify the author to the corresponding affiliation, 1, 2, 3, etc. should be inserted.



\runningtitle{Efficient Parametric FE Analysis of Passive Microwave Devices Using a DD - MOR Approach}

\runningauthor{O. Farle et al.}

\correspondence{M. L\"osch\\ (m.loesch@lte.uni-saarland.de)}



\received{}
\pubdiscuss{} %% only important for two-stage journals
\revised{}
\accepted{}
\published{}

%% These dates will be inserted by the Publication Production Office during the typesetting process.


\firstpage{1}

\maketitle



\begin{abstract}
In this paper we present a new methodology for efficient parametric finite element analysis.
Since broadband parameter variations demand for increased numerical robustness,
we propose the use of a multipoint order reduction method.
These come along with high computational costs for constructing the projection space.
Since most of the considered parameters are local,
this means only a small part of the field domain changes,
the generation process of the reduced order model can be accelerated by incorporating a domain decomposition approach.
We present several examples that demonstrate the efficacy of the proposed technique.
\end{abstract}

\introduction
The finite element (FE) method is very well known for its flexibility and reliability
as well as its relatively high computational times.
%This is a problem when frequency sweeps and material parameter sweeps have to be accounted for
%and, indeed, several techniques have been proposed to get faster sweeps in frequency \cite{Bracken98} %\cite{Pol97}
%and,
%in material parameters \cite{Far06,FarIP}.
This can become a major issue,
if the solution is sought as a function of a multidimensional parameter space.
Possible applications are optimization scenarios or parametric studies.
Typical parameters include frequency \cite{Bracken98},
material properties \cite{Far06} and geometry parameters \cite{Daniel2004}.
A promising candidate for efficiently handling these parameter dependent systems is model order reduction (MOR) \cite{Antoulas2005}.
Almost all of the modern MOR approaches are projection based.
This means,
the original system,
with up to millions of degrees of freedoms,
is projected onto a low dimensional subspace of global shape functions,
thus drastically reducing the dimension of the model with introducing only a minor additional approximation error.
The number of global shape functions is typically between ten and a few hundred.
The various order reduction methods mainly differ in the projection spaces
and the way the projection matrices are computed numerically.
MOR algorithms can either be characterized as a single point or a multipoint method.
Most single point methods \cite{Gunupudi2002,Daniel2004} are of moment matching type.
This means,
the resulting reduced order model (ROM) matches the derivatives of the transfer function of the full model
at a given expansion point up to a certain order.
Since ROM and original model share the same parameterized structure,
the ROM is a multidimensional Pad� approximation \cite{Weile1999}.
The key feature of single point methods is the fact,
that the system matrix of the full model only has to be factorized in the expansion point,
so one matrix factorization is sufficient.
This advantage looses importance as soon as iterative solvers have to be used.
Although in recent time,
progress has been made in improving the numerical stability of single point methods for the multi parameter case
\cite{Codecasa2005,Feng2005,FarIP},
multipoint methods like \cite{Weile2001,PrudHomme02} are still numerically more robust.
Additionally they usually result in ROMs of smaller dimension,
as demonstrated in \cite{Schultschik2008} for the single parameter case.
The main drawback of multipoint approaches
is the need for solving the full FE system at several points in the parameter space.
This is especially true for the method of proper orthogonal decomposition (POD) \cite{Holmes96},
where the projection basis is subsequently reduced by a low rank approximation,
to achieve a further decrease in ROM dimension.


\section{HEADING}
TEXT

\subsection{HEADING}
TEXT

\subsubsection{HEADING}
TEXT




\conclusions
%% \conclusions[modified heading if necessary]
TEXT




%\appendix
%\section{\\ \\ \hspace*{-7mm} HEADING}    %% Appendix A
%
%\subsection                               %% Appendix A1, A2, etc.




\begin{acknowledgements}
TEXT
\end{acknowledgements}


\bibliographystyle{copernicus}
\bibliography{IEEEabrv,Kleinheubach_v0.0}
%\begin{thebibliography}{}
%
%\bibitem[AUTHOR(YEAR)]{LABEL}
%REFERENCE 1
%
%\bibitem[AUTHOR(YEAR)]{LABEL}
%REFERENCE 2
%
%...
%
%\end{thebibliography}


%% Literature citations
%% command                        & example result
%% \citet{jones90}|               & Jones et al.\ (1990)
%% \citep{jones90}|               & (Jones et al., 1990)
%% \citep{jones90,jones93}|       & (Jones et al., 1990, 1993)
%% \citep[p.~32]{jones90}|        & (Jones et al., 1990, p.~32)
%% \citep[e.g.,][]{jones90}|      & (e.g., Jones et al., 1990)
%% \citep[e.g.,][p.~32]{jones90}| & (e.g., Jones et al., 1990, p.~32)
%% \citeauthor{jones90}|          & Jones et al.
%% \citeyear{jones90}|            & 1990






%%% FIGURES %%%%%%%%%%%%%%%%%%%%%%%%%%%%%%%%%%%%%%%%%%%%%%%%%%%%%%%%%%%%%%%%%%%%
%
%
%%% ONE-COLUMN FIGURES
%
%%f
%\begin{figure}[t]
%\vspace*{2mm}
%\begin{center}
%\includegraphics[width=8.3cm]{FILE NAME}
%\end{center}
%\caption{TEXT}
%\end{figure}
%
%
%
%%% TWO-COLUMN FIGURES
%
%%f
%\begin{figure*}[t]
%\vspace*{2mm}
%\begin{center}
%\includegraphics[width=12cm]{FILE NAME}
%\end{center}
%\caption{TEXT}
%\end{figure*}
%
%
%%% TABLES %%%%%%%%%%%%%%%%%%%%%%%%%%%%%%%%%%%%%%%%%%%%%%%%%%%%%%%%%%%%%%%%%%%%
%
%
%%% ONE-COLUMN TABLE
%
%%t
%\begin{table}[t]
%\caption{TEXT}
%\vskip4mm
%\centering
%\begin{tabular}{column = lcr}
%\tophline
%
%\middlehline
%
%\bottomhline
%\end{tabular}
%\end{table}
%
%
%%% TWO-COLUMN TABLE
%
%%t
%\begin{table*}[t]
%\caption{TEXT}
%\vskip4mm
%\centering
%\begin{tabular}{column = lcr}
%\tophline
%
%\middlehline
%
%\bottomhline
%\end{tabular}
%\end{table*}
%
%
%%% The different columns must be seperated with a & command and should
%%% end with \\ to identify the column brake.
%
%%%%%%%%%%%%%%%%%%%%%%%%%%%%%%%%%%%%%%%%%%%%%%%%%%%%%%%%%%%%%%%%%%%%%%%%%%%%%%%
%
%
%%% If figures and tables must be numbered 1a, 1b, etc. the following command
%%% should be inserted before the begin{} command.
%
%\addtocounter{figure}{-1}\renewcommand{\thefigure}{\arabic{figure}a}


\end{document}
