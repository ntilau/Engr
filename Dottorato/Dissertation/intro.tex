% !TEX root = ln_diss.tex
\chapter{Introduction} \label{chap:INT}

Nowadays, several successful commercial packages for solving electromagnetic problems are available. Such computational electromagnetics software is typically based on one or more \quotes{traditional} rigorous (full-wave) techniques such as the finite differences time-domain \cite{CSTMWS, SEMCADX} and the finite element method \cite{ComsolRF, HFSS}, which are differential based methods, or the method of moments \cite{FEKO} which is integral based, and sometimes include some of their hybridizations \cite{Garg2008, Davidson2011, RIB2011}. The robust formulations they implement have been validated throughout decades by multiple physical measurements of real-life applications, up to the point they form an invaluable part of current radio frequency and microwave engineering practice. Without these computational modeling methods, probably many highly technological applications would not have been realized yet.

However, many challenging applications still remain to be tackled due to the limited availability of computational resources. The higher is the electrical size of the problem, even if quite geometrically simple, the higher becomes the number of unknowns required to compute the fields and other related parameters. Some of these are large antenna arrays, antenna platform positioning problems, radar cross section of electrically large targets and with composite materials, terahertz and optical devices. The problem size, in terms of unknowns $N$ necessary for an acceptable (error controlled) analysis at a frequency $f$, typically scales as $N \propto O(f^3)$ for differential based methods and as $N \propto O(f^2)$ for integral methods \cite{lee2011maturity}. The same behavior is encountered for geometrically complicated models. Even if smaller than the wavelength, better accuracy is needed around the conductors corners and at interfaces between materials, some of the known sources of field singularities. Some of these applications are the frequency selective surfaces used as electromagnetic coatings and the signal integrity computation in high frequency integrated circuits \cite{lee2011maturity, mittra2004look}. A combination of geometrical and electromagnetic size complexities can be found in recent nano-optical applications, frequencies at which metals behave as high permittivity materials due to surface plasmons surrounded by the unitary permittivity of the air \cite{tsukerman2008computational}. Accurate solution of these kind of problems is still an active research field \cite{tsukerman1998comparison}.

Among all the problems that still remain to be solved, those comprehending nonlinear materials are still currently faced by the computational electromagnetics community, principally by the use of finite differences time-domain schemes which allow for straightforward implementation of nonlinearities \cite{goorjian1992direct, ziolkowski1993full, ziolkowski1994nonlinear, joseph1997fdtd, teixeira2008time, sasanpour2010analysis, potravkin2012numerical, frances2013split}. The main investigated fields of application initially were at optical frequencies, including harmonic generation, nano-plasmonics and solitons propagation in Kerr-like media. In fact, very high intensity fields are typically generated at those frequencies by lasers or even light emitting diodes, and hence materials that at microwave frequencies may behave as linear cannot anymore be accurately handled at optical ones with linear solvers. At lower frequencies, for the solution magneto-quasi-static problems like the evaluation eddy currents in ferromagnetic materials, a method using finite elements was early introduced by Silvester in 1970 \cite{silvester1970finite}. A time-harmonic scheme was there presented, allowing for fast computation of steady state fields. However, information on the distortion introduced by permeability saturation was still neglected. Years later, Yamada et al. \cite{yamada1988harmonic, yamada1989harmonic, yamada1991calculation} introduced a first multi-harmonic scheme, the harmonic balance finite element method, allowing for accurate treatment of nonlinearities \cite{gyselinck2002harmonic, pascal2003coupling, zhao2010analysis, zhao2011analysis}. Contemporaneously, finite element time domain schemes where introduced to allow transient analyzes \cite{biro1995various}. However, due to the immediate extrapolation of steady state fields which indeed are almost always sought for, a frequency-domain scheme often results to be preferable. It still appears that no multi-harmonic schemes have been employed from microwaves to optical frequencies.


\section{Efficient solvers for computational electromagnetics}

Even if they perform worse in terms of electrical size, differential based methods, leading to sparse matrices, can be directly solved with an asymptotic complexity of $O(N^2)$ \cite{gupta1997design} while the dense matrices of integral methods can be directly inverted with $O(N^3)$ complexity \cite{Cormen2009}. Furthermore, linear solvers (either stationary or not) behave dramatically worse as $N$ grows, thus require a particular attention to preconditioning in order to restore their performances. Three main approaches have been successfully adopted to steer the computational complexities of electromagnetic solvers down to linear or $O(N\log N)$:
\begin{itemize}
\item \textit{Multigrid methods}, which to some extent are based on multiple superimposed discretization levels, where the information derived from a coarse level (where a direct solver performs better) can be used to accelerate the computations on a finer level with a linear solver. This hierarchical decomposition of accuracy levels have been successfully employed for differential based methods \cite{Briggs2000, hill2003stabilized, ingelstrom2004higher, gheorghe2005multigrid, zhu2006multigrid, hill2006schnelle, ingelstrom2007comparison}. The integral based multigrid methods, referred as \quotes{multiresolution methods} \cite{vipiana2005multiresolution, khorrami2010fast}, are not yet affirmed as their differential counterparts.
\item \textit{Domain decomposition methods}, which follows the \textit{divide et impera} strategy, dividing a wide fine grid into smaller parts where the field solutions can be easily computed, then proper transmission conditions or subdomains coupling strategy have to be enforced in order to recover the fields within the whole domain \cite{quarteroni1999domain, toselli2005domain, mathew2008domain} . These methods have been extensively studied and employed to solve Maxwell's equations \cite{li2006vector, sun2005nonconforming, vouvakis2007domain, rodriguez2006new, zhao2006solving, barka2007domain, zhao2008domain, rawat2008domain, grasedyck2009domain, rawat2009finite, ozgun2010iterative, peng2010non, takei2010full, peng2011integral, shao2011full, wang2012domain,widlund2007domain} for both differential and integral methods (or their hybridization) up to tens of millions of unknowns with common personal computers.
\item \textit{Multipole methods}, which fundamentally group local method of moments near field solutions into multipoles that allow to compute far field couplings between the groups. These methods are intrinsically related to integral based methods, where dipole or multipoles can be accurately handled. They are known as the \textit{fast multipole method} or \textit{multilevel fast multipole algorithm} \cite{Chew2001, Gumerov2004} for integral equations.
\end{itemize}
In many cases, several difficulties remain for all the aforementioned methods. Typically based on some precise assumptions for which the iterative solvers they employ should converge, the end-user of the method still have to be highly skilled and experienced in order to properly conduct the analyzes. A lot of efforts still have to be done in order to achieve robust solvers, especially from the mathematics behind the implemented code.

Also, several improvements have been achieved to perform fast parameter sweeps. Once some solutions, for some parameter values, are computed by one of the previous \quotes{traditional} or efficient methods, fast intermediate solutions can be obtained by proper interpolation schemes. These are known as the \textit{model order reduction} methods \cite{farle2006multivariate, ntibarikure2012model} for differential based methods or as the \textit{characteristic basis functions} methods \cite{prakash2003characteristic} for integral based methods. Several parameter dependent solutions of full-wave methods are collected and orthonormalized with some spectral decomposition (Gram-Schmidt) or singular value decomposition. These solution vectors are then used to expand large-scale solutions on which the orginal problem is projected. Finally, if these solutions constitute a basis for the whole function space, then a few operations are sufficient to perform a parameter sweep with a controlled order of accuracy. The parameters can be the frequency of analysis, material properties, complex excitation amplitudes of a multiport device and many others. In principle, the large cost of multiple solutions computation in a parameter sweep is truncated once a good basis is found.

Among the \quotes{traditional} methods, differential based methods allow a straightforward treatment of materials properties and, for the finite element method, a better discretization of geometrical bodies. Integral methods are better suited for open problems, where the differential based methods require approximative boundary truncation techniques to implement Sommerfeld's radiation condition: absorbing boundary condtions \cite{Bayliss1982, Silvester1988} or perfectly matched layers (PMLs) \cite{Berenger1994}. However, many successful hybridizations of differential based and integral based methods have been reported \cite{xu1997hybrid, zhao2011efficient}. It is worth noticing that the introduction of integral equations has a significant drawback in reducing the computational efficiency \cite{zhao2006solving}. This has led to the choice of the finite element method as core development for efficient schemes that will be analyzed throughout this dissertation.

\section{A reason for nonlinear analyzes at microwaves}

It is well known that polycrystalline magnetic oxides like ferrites or other ferromagnetic compounds such the yttrium iron garnet, for their anisotropy, can be used to realize non-reciprocal microwave devices such as circulators, isolators and phase shifters \cite{pozar1998microwave, adam2002ferrite}. It is also known that these devices are typically critical when dealing with high power electromagnetic fields, due to the spurious fields they induce \cite{suhl1956nonlinear, suhl1957theory, bailey1979study}. Ferromagnetic materials were probably the first to present a nonlinear behavior at microwave frequencies. Very few attempts to predict the spurious fields generated in such devices have been reported in literature \cite{wu1976study, how1997nonlinear}.

Known as \textit{passive intermodulation}, the nonlinearities cause a critical limit in the design of microwave systems \cite{sanford1993passive} and their calculation methods are still very limited. This phenomenon is also imparted to metal contacts \cite{arazm1980nonlinearities}, metallic wires and dielectric cables \cite{amin1978coaxial} and particularly to metals oxidation \cite{bond1979intermodulation}. In a general form, these can be viewed as nonlinear electromagnetic properties of materials, that is field dependent permittivities, permeabilities and conductivities.

Also, during the last two decades, due to several improvements in the field of digital electronics, when seeking for thin-films materials with high permittivities to implement the capacitances in dynamic random access memories \cite{scott1994dielectric, shaw1999effect}, barium strontium titanate compounds have demonstrated a noticeable nonlinear behavior at microwave frequencies \cite{kozyrev1997ferroelectric, mateu2006measurements, mateu2007frequency, giere2007characterization}. Their electro-optic effect \cite{takeda2010dielectric}, principally due to a second order or Kerr-like permittivity, can hence be exploited to design tunable capacitors \cite{sigman2008voltage}, or in general to control the characteristics of any microwave device that use this kind of materials.

Finite elements can considerably help in the computation of nonlinear phenomena products, especially for its capability to easily handle material properties and geometries. Once again, it is the best suited computational electromagnetic method for nonlinear analyzes.

\section{Contributions of this dissertation}

The chosen fields of research have led to the implementation of a finite element software, namely \quotes{FES}. Several formulations have been implemented in FES, basically based on the application of the Galerkin framework on both two- and three-dimensional domains. The two-dimensional package, FES-2D, mainly coded in the high level Matlab$^\text{\circledR}$~language, was prevalently used to assess the formulations, which might result to be excessively time demanding to implement in lower level of abstraction languages. Then, the transfer of the formulations to tree-dimensional problems in FES-3D could be done relatively faster in an objective paradigm C++ code. FES-3D, using several third party, open source\footnote{Almost all the employed third party codes are at least provided with the in the Lesser General Public Licence, and some where totally free for reuse.}, codes, some written in Fortran or C, has been compiled with the GNU GCC 4.8.1 on {x86}\textunderscore{64} architectures with the -Ofast optimization flag enabled.

The present dissertation is structured as follows:
\begin{itemize}
\item The second chapter introduces to the electromagnetic radiation mechanism, which is known since more than a century to be governed by a set of partial differential equations collected into Maxwell's equations. A wave equation for the electric field is then derived, allowing for single partial differential equation solution. It is well known that this equation can be accurately solved by numerical methods as long as the frequency of analysis is above the lower limit that causes badly conditioned matrices, which is the case for all the experiments conducted here. The Galerkin framework is then introduced, with the necessary mathematical background to understand its efficacy. Finally, the formulations employed to solve, later on, several waveguide and radiation problems are introduced. Within this phase, the FES results are compared to commercial models analyzes for validation purposes.
\item In the third chapter, the domain decomposition concept is introduced with two different schemes, the Schur complement and the finite element tearing and interconnecting. Both the methods are tested on the simple case of a rectangular waveguide. Preconditioners for Krylov subspace methods are then built on a domain decomposition scheme and the convergence behavior is analyzed extensively. Finally, the numerical complexity of the method is derived to ensure its applicability to very large problems.
%a large frequency selective radome-enclosed patch antennas array is analyzed, showing the capabilities of the methods. 
\item In the fourth chapter, the first known attempt to apply the harmonic balance finite element method to microwave problems is presented. Almost all passive nonlinear problems require steady state computations and hence the method results to be very well suited. Several test cases on two-dimensional problems are shown to explain the method. One of the previous domain decomposition schemes have also been employed to accelerate the nonlinear analyzes. A three-dimensional barium strontium titanate based test case is shown, somehow proving the capabilities of the method.
\item Finally, some conclusions will be drawn in the last chapter. The analyzed and implemented methods open the way to many, still unperformed, analyzes. A possible outlook, matter of emerging technologies, will be hence discussed.
\end{itemize}

