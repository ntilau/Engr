\phantomsection
\addcontentsline{toc}{chapter}{Conclusione}
\chapter*{Conclusione}


\par I passi avanti della tecnologia, come le tecniche di realizzazione sempre pi� efficienti e rapide, unite
alla ricerche di mercato mirate, hanno portato alla creazione di prodotti - in particolare
nell'elettronica di consumo - che possono essere definiti di ``massa'', nel vero senso della parola.
L'elettronica infatti, dai cellulari, ai personal computer fino alle videocamere, viene prodotta in
quantit� industriale e questo ha reso possibile anche l'abbattimento dei costi e quindi la diffusione
capillare di questi prodotti. Il sistema che ruota intorno a questa produzione richiede prodotti in continua e rapida evoluzione, tale da mantenere sempre in
fermento il mercato. Per questo, come gi� affermato, il time to market del prodotto si � notevolmente
ridotto, ma, al tempo stesso sono aumentate le richieste da soddisfare prima della messa in
commercio, come la completa verifica del sistema (comprensiva di misure e test funzionali, controlli di qualit� e di affidabilit�). Risulta quindi impensabile progettare un qualsiasi sistema elettronico (anche semplice, come nelle nostre esercitazioni) senza l'ausilio di un software di progettazione assistita da calcolatore (CAD). Anzi, i CAD stessi, seguendo il trend di continua evoluzione del mercato, sono diventati sempre pi� completi (come il nostro ADS), in grado di seguire una progettazione completa del circuito in ogni singolo aspetto: schematico, layout, interconnessioni a livello hardware, simulazioni nel dominio del tempo e della frequenza, simulazioni elettromagnetiche ed addirittura simulazioni
meccaniche e termocinetiche in alcuni casi. Dunque, oggigiorno, un progettista non pu� prescindere dalla conoscenza di questi  strumenti, e come ci ha insegnato questo corso, deve riuscire ad entrare il pi� velocemente possibile nei meccanismi del simulatore e, possibilmente conoscerne diversi, in modo da sfruttare quello pi� opportuno a seconda del problema che deve affrontare.
%P.S.: Prima stesura da rileggere e rimaneggiare�dimmi se va bene come linea guida senn� �inutile�ola�
